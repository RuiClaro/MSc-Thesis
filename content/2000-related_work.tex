% -*- coding: utf-8 -*-
%

% reset all acronym expansions
\acresetall

%%%%%%%%%%%%%%%%%%%%%%%%%%%%%%%%%%%%%%%%%%%%%%
\chapter{Related Work}
\label{ch:RelatedWork}
%%%%%%%%%%%%%%%%%%%%%%%%%%%%%%%%%%%%%%%%%%%%%%


This Chapter provides an overview of the concepts relevant to privacy-preserving data processing.
We start by defining Data Security and Data Privacy, in sections \ref{sec:DataSecurity} and \ref{sec:DataPrivacy} respectively, and describe the differences between them.
Section \ref{sec:PrivacyImplicationsPersonalDataProcessing} presents the concepts of data processing and \ac{dm}, gives an overview of the \ac{crisp} model and defines the attack models that can be assumed when developing a \ac{ppdm} solution.
For developing a \ac{ppdm} algorithm, we can implement one or more of the privacy-preserving techniques briefly presented in Section \ref{sec:PrivacyPreservingTechniques}. 
We present \ac{ppml} in Section \ref{sec:PrivacyPreservingMachineLearning}. Finally, in Section \ref{sec:UseCases} we discuss use cases that show what can be achieved.

  %%%%%%%%%%%%%%%%%%%%%%%%%%%%%%%%%%%%%%%%%%%%%%%%%%%%%%%%%%%%%%%%%%%%%%%%%%%%%
  %
%%%%%                        THE BEGINNING
 %%%
  %

\input content/2010-data_security.tex
\input content/2020-data_privacy.tex
\input content/2030-data_process.tex
\input content/2040-techniques.tex
\input content/2050-ppml.tex
\input content/2060-use_cases.tex
  %%%%%%%%%%%%%%%%%%%%%%%%%%%%%%%%%%%%%%%%%%%%%%%%%%%%%%%%%%%%%%%%%%%%%%%%%%%%%
  %
%%%%%                        LAST SECTION
 %%%
  %

%%%%%%%%%%%%%%%%%%%%%%%%%%%%%%%%%%%%%%%%%%%%%%%%%%%
\section{Summary}
\label{sec:SummaryRelatedWork}
%%%%%%%%%%%%%%%%%%%%%%%%%%%%%%%%%%%%%%%%%%%%%%%%%%%

This Chapter provided an overview of the state of the art surrounding privacy-preserving data processing.
We started by defining Data Security and Data Privacy in Sections \ref{sec:DataSecurity} and \ref{sec:DataPrivacy}.
We described the concept of \ac{dm} and data processing, in Section \ref{sec:PrivacyImplicationsPersonalDataProcessing}.
In Section \ref{sec:PrivacyPreservingTechniques} we described privacy-preserving techniques that can be used to implement a \ac{ppml} algorithm.
We presented \ac{ppml} in Section \ref{sec:PrivacyPreservingMachineLearning}. Finally, we discussed known use cases that illustrate the benefits of what can be achieved in the field in Section \ref{sec:UseCases}.


  %
 %%%
%%%%%                        THE END
  %
  %%%%%%%%%%%%%%%%%%%%%%%%%%%%%%%%%%%%%%%%%%%%%%%%%%%%%%%%%%%%%%%%%%%%%%%%%%%%%
