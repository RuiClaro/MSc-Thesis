
%%%%%%%%%%%%%%%%%%%%%%%%%%%%%%%%%%%%%%%%%%%%%%%%%%%
\section{Use Cases}
\label{sec:UseCases}
%%%%%%%%%%%%%%%%%%%%%%%%%%%%%%%%%%%%%%%%%%%%%%%%%%%


In terms of privacy-preserving Machine Learning and its applications, it is important to distinguish the context of what is being processed. The information contained in the data has to be treated differently depending on what that data contains. Different data can be subject to different constraints regarding laws and privacy. Some sensitive data may be only processable in a local environment, while other data can only be processed in a less individualized way.

We now detail three different subjects that are of importance in the area and are subject to different privacy constraints.

\begin{itemize}
	\setlength\itemsep{1em}
	\item \textbf{Health records:} The healthcare system is one of the examples where vast amounts of data are collected every day, and it is of relevance to do Data Mining on patient records, for a better understanding of patients and to improve the healthcare system. But patient records contain very sensitive information about individuals and cannot be processed without the Data Mining system being in compliance with the legislation on Data Privacy, therefore it is of interest to build privacy-preserving systems for the healthcare system, so that hospitals and other health-related organizations can share and infer knowledge without violating the privacy of their patients.

	\item \textbf{Students and taxes:} Bogdanov \textit{et al.} \cite{Bogdanov2015} made in 2015 a statistical study using \ac{SMPC} to look for correlations between working during university studies and failing to graduate in time. For this study, it was necessary to link the database of individual tax payments and the database of higher education universities. These types of government data are subject to strict legislation and cannot simply be handled without strong privacy guarantees.

    To solve this problem, a \ac{SMPC} system was developed and deployed that could assure a level of privacy that would be in compliance with the laws on Data Privacy. The data processing steps were all made using \ac{SMPC} between three parties, using \ac{OT} so that each party would not know each other inputs.

    In the end, the study using \ac{SMPC} was compared with an anonymized study using 3-anonymity. The loss of samples in the latter was 10\%-30\%, depending on the demographic group, thus suggesting that producing studies on existing databases using \ac{SMPC} to enforce privacy can give more accurate results than the same study run using \textit{k}-anonymity measures.

	\item \textbf{Human mobility:} Another subject that provides great challenges in the field of Data Privacy are the mobility traces generated by people when driving, walking, etc. Mobility traces are highly unique so it is possible, even after anonymizing the dataset, to link an individual to his mobility patterns, as shown by Montjoye \textit{et al.} \cite{de2013unique}. Since mobility data contains the approximate whereabouts of individuals, it can be used to reconstruct their movements across space and time. Applying privacy-preserving techniques to process this highly sensitive data can result in better geographic-based recommendation systems.
\end{itemize}


%%%%%%%%%%%%%%%%%%%%%%%%%%%%%%%%%%%%%%%%%%%%%%%%%%%%%%%%%%%%%%%%%%%%%%%%%%%%%%%%%%%%%%%%%%%%%%%%%%%%%%%