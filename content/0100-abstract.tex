% -*- coding: utf-8 -*-
%

%%%%%%%%%%%%%%%%%%%%%%%%%%%%%%%%%%%%%%%%%%%%%%%%%%%%%%%%%%%%%%%%%%%%%%%%%%%%
\chapter*{Resumo} % Resumo do documento (em Português), tradução do Abstract
                  % Mínimo 300 palavras
%%%%%%%%%%%%%%%%%%%%%%%%%%%%%%%%%%%%%%%%%%%%%%%%%%%%%%%%%%%%%%%%%%%%%%%%%%%%
\thispagestyle{empty}

% Portuguese translation of the abstract


\newpage

%%%%%%%%%%%%%%%%%%%%%%%%%%%%%%%%%%%%%%%%%%%%%%%%%%%%%%%%%%%
\chapter*{Abstract} % Abstract of the document (in English)
                    % Minimum 300 words
%%%%%%%%%%%%%%%%%%%%%%%%%%%%%%%%%%%%%%%%%%%%%%%%%%%%%%%%%%%
\thispagestyle{empty}

%#########NEEDS REVIEW###############
We live in the age of Big Data. Personal user data, in particular, is necessary for the operation and improvement of everyday Internet services like Google, Facebook, WhatsApp, Spotify, etc. Many times, the capture and use of personal data is not made explicit to the users, but it is central to the business model of companies. However, each person’s right to privacy has to be respected. How can these two conflicting needs be reconciled, \textit{i.e.}, how can we build useful Big Data systems that are respectful of user privacy?
The goal of this work is to design and implement a proof-of-concept of a platform for performing privacy preserving computations, providing an easy-to-use method to implement privacy-preserving techniques. This system could be used, for example, to monitor the vital signs of patients (without exposing them to other people), to produce real time recommendations based on location (without disclosing location to others), sports/fitness applications, etc.
This proof-of-concept will implement privacy-preserving versions of Machine Learning algorithms and compare them against a baseline reference. \commentPT{so the trade-offs can be quantified and better understood.}

\newpage
