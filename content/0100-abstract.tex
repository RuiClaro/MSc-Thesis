% -*- coding: utf-8 -*-
%

%%%%%%%%%%%%%%%%%%%%%%%%%%%%%%%%%%%%%%%%%%%%%%%%%%%%%%%%%%%%%%%%%%%%%%%%%%%%
\chapter*{Resumo} % Resumo do documento (em Portugu\^{e}s), tradu\c{c}\~{a}o do Abstract
                  % M\'{i}nimo 300 palavras
%%%%%%%%%%%%%%%%%%%%%%%%%%%%%%%%%%%%%%%%%%%%%%%%%%%%%%%%%%%%%%%%%%%%%%%%%%%%
\thispagestyle{empty}

Vivemos na era do \textit{Big Data}. Os dados pessoais dos utilizadores, em particular, s\~{a}o necess\'{a}rios para o desenvolvimento, funcionamento e melhoria constante dos servi\c{c}os dispon\'{i}veis na Internet, nomeadamente o Google, Facebook, WhatsApp, Spotify, entre tantos outros. Muitas vezes, a recolha e o uso dos dados pessoais n\~{a}o s\~{a}o expl\'{i}citos para os utilizadores, embora a sua utiliza\c{c}\~{a}o seja central para o modelo de neg\'{o}cios destas empresas. No entanto, o direito \`{a} privacidade de cada indiv\'{i}duo tem de ser respeitado.

De que forma podem estas duas necessidades conflituantes serem reconciliadas, ou seja, como podemos construir sistemas de \textit{Big Data} que respeitem a privacidade do utilizador? O objetivo deste trabalho \'{e} desenhar e implementar uma ``prova de conceito'' de uma plataforma, para realizar computa\c{c}ões que preservem a privacidade dos utilizadores. Pretende-se disponibilizar, deste modo, um m\'{e}todo de f\'{a}cil utiliza\c{c}\~{a}o para a implementa\c{c}\~{a}o de t\'{e}cnicas de preserva\c{c}\~{a}o da privacidade. Este sistema poderia ser utilizado, por exemplo, para monitorizar os sinais vitais dos pacientes (sem os expor a outras pessoas), para produzir recomenda\c{c}ões em tempo real com base na localiza\c{c}\~{a}o (mas sem a divulga\c{c}\~{a}o da mesma), para aplica\c{c}ões de desporto, entre outros. Assim, esta ``prova de conceito'' visa implementar versões de algoritmos de aprendizagem autom\'{a}tica que preservem a privacidade, e que, ao compar\'{a}-los com uma refer\^{e}ncia, permitam uma melhor compreens\~{a}o das rela\c{c}ões e benef\'{i}cios criados com o uso desta tecnologia.


\newpage

%%%%%%%%%%%%%%%%%%%%%%%%%%%%%%%%%%%%%%%%%%%%%%%%%%%%%%%%%%%
\chapter*{Abstract} % Abstract of the document (in English)
                    % Minimum 300 words
%%%%%%%%%%%%%%%%%%%%%%%%%%%%%%%%%%%%%%%%%%%%%%%%%%%%%%%%%%%
\thispagestyle{empty}

We live in the age of Big Data. Personal user data, in particular, are necessary for the operation and improvement of everyday Internet services like Google, Facebook, WhatsApp, Spotify, etc. Many times, the capture and use of personal data are not made explicit to the users, but they are central to the business model of companies. However, each individual's right to privacy has to be respected.

How can these two conflicting needs be reconciled, i.e., how can we build useful Big Data systems that are respectful of user privacy?
The goal of this work is to design and implement a ``proof-of-concept'' of a platform for performing privacy-preserving computations, providing an easy-to-use method to implement privacy-preserving techniques. This system could be used, for example, to monitor patient's vital signs (without exposing them to other people), to produce real-time recommendations based on location (without disclosing the location to others), sports/fitness applications, etc.
This proof-of-concept will implement privacy-preserving versions of \ac{ml} algorithms and compare them against a baseline reference, allowing a better understanding of the trade-offs of using this technology.

\newpage
