% -*- coding: utf-8 -*-
%

% reset all acronym expansions
\acresetall

%%%%%%%%%%%%%%%%%%%%%%%%%%%%%%%%%%%%%%%%%%%%%%
\chapter{Implementation}
\label{ch:Implementation}
%%%%%%%%%%%%%%%%%%%%%%%%%%%%%%%%%%%%%%%%%%%%%%

When building a platform in the scope of privacy-preserving \ac{ml}, we must consider not only the traditional steps in data processing mentioned in section \ref{sec:PrivacyImplicationsPersonalDataProcessing} and in Figure \ref{fig:crisp-dm}, but also have an increased care when preprocessing data to incorporate the cryptographic techniques that we used.
This chapter describes the work that was done in implementing a privacy-preserving \ac{ml} platform. We start off with a description of the datasets chosen to evaluate our platform \ref{sec:DatasetsImplementation}. Then, we explain the preprocessing that was done to those datasets \ref{sec:DataPreProcessingImplementation}.
Section \ref{sec:BaselineImplementation} presents the baseline implementation of the chosen \ac{ml} algorithms, resorting to a widely used \ac{ml} toolkit for Python.
In section \ref{sec:ExpandedAlgorithmsImplementation} the implementation of the prediction phase of the algorithms is detailed.
In section \ref{sec:CryptoDomainImplementation}, we detail which cryptographic protocols we used, why, and how we implemented them, mentioning which toolkits were used.
For understanding the uses of the solution developed, we detail in section \ref{sec:usecaseHealthcare} a use case our implementation, and finally in section \ref{sec:components} we present a possible usage for our implementation, in the form of a toolkit that could be used in a business environment.


  %%%%%%%%%%%%%%%%%%%%%%%%%%%%%%%%%%%%%%%%%%%%%%%%%%%%%%%%%%%%%%%%%%%%%%%%%%%%%
  %
%%%%%                        THE BEGINNING
 %%%
  %


\input content/3010-implementation.tex
\input content/3020-use_case.tex
\input content/4040-api.tex


  %%%%%%%%%%%%%%%%%%%%%%%%%%%%%%%%%%%%%%%%%%%%%%%%%%%%%%%%%%%%%%%%%%%%%%%%%%%%%
  %
%%%%%                        LAST SECTION
 %%%
  %

  
%%%%%%%%%%%%%%%%%%%%%%%%%%%%%%%%%%%%%%%%%%%%%%
\section{Summary}
\label{sec:SummaryImplementation}
%%%%%%%%%%%%%%%%%%%%%%%%%%%%%%%%%%%%%%%%%%%%%%


In this chapter, we discussed the implementation of a privacy-preserving \ac{ml} platform. We started by describing the datasets chosen to evaluate the platform in section \ref{sec:DatasetsImplementation}, and then we explained the preprocessing step in section \ref{sec:DataPreProcessingImplementation}.
Section \ref{sec:BaselineImplementation} presented the implementation of the baseline for the chosen \ac{ml} algorithms, and section \ref{sec:ExpandedAlgorithmsImplementation} detailed the implementation of the evaluation phase of those algorithms.
In section \ref{sec:CryptoDomainImplementation}, we detailed which cryptographic protocols we used, why, and how we implemented them, mentioning which toolkits were used.
For understanding the uses of the solution developed, we detailed in section \ref{sec:usecaseHealthcare} a use case our implementation, and finally in section \ref{sec:components} we presented a possible usage for our implementation, in the form of a toolkit that could be used in a business environment.


  %
 %%%
%%%%%                        THE END
  %
  %%%%%%%%%%%%%%%%%%%%%%%%%%%%%%%%%%%%%%%%%%%%%%%%%%%%%%%%%%%%%%%%%%%%%%%%%%%%%
