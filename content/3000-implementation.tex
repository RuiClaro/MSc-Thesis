% -*- coding: utf-8 -*-
%

% reset all acronym expansions
\acresetall

%%%%%%%%%%%%%%%%%%%%%%%%%%%%%%%%%%%%%%%%%%%%%%
\chapter{Implementation}
\label{ch:Implementation}
%%%%%%%%%%%%%%%%%%%%%%%%%%%%%%%%%%%%%%%%%%%%%%

When building a platform for \ac{ppml}, we must not only consider the traditional steps in data processing mentioned in Section \ref{sec:PrivacyImplicationsPersonalDataProcessing} and in Figure \ref{fig:crisp-dm}, but also have an increased care when preprocessing data to incorporate the cryptographic techniques.

This Chapterdescribes the work that was done in implementing a \ac{ppml} platform. In Section \ref{sec:DatasetsImplementation}, we start off with a description of the datasets chosen to test our platform. Then, we explain the preprocessing that was done to those datasets, in Section \ref{sec:DataPreProcessingImplementation}.
Section \ref{sec:BaselineImplementation} presents the baseline implementation of the chosen \ac{ml} algorithms, resorting to a widely used \ac{ml} toolkit for Python.
In Section \ref{sec:ExpandedAlgorithmsImplementation} we detail the implementation of the prediction phase of the algorithms.
In Section \ref{sec:CryptoDomainImplementation}, we present the cryptographic protocols used, why we used them, and how we implemented them, mentioning which toolkits were used.
For understanding the applicability of the solution developed, we detail in Section \ref{sec:usecaseHealthcare} a use case for our implementation. Finally in Section \ref{sec:components} we present a possible usage for our implementation, in the form of a toolkit that could be used in a business environment.


  %%%%%%%%%%%%%%%%%%%%%%%%%%%%%%%%%%%%%%%%%%%%%%%%%%%%%%%%%%%%%%%%%%%%%%%%%%%%%
  %
%%%%%                        THE BEGINNING
 %%%
  %


\input content/3010-implementation.tex
\input content/3020-use_case.tex
\input content/4040-api.tex


  %%%%%%%%%%%%%%%%%%%%%%%%%%%%%%%%%%%%%%%%%%%%%%%%%%%%%%%%%%%%%%%%%%%%%%%%%%%%%
  %
%%%%%                        LAST SECTION
 %%%
  %

  
%%%%%%%%%%%%%%%%%%%%%%%%%%%%%%%%%%%%%%%%%%%%%%
\section{Summary}
\label{sec:SummaryImplementation}
%%%%%%%%%%%%%%%%%%%%%%%%%%%%%%%%%%%%%%%%%%%%%%


In this Chapter, we discussed the implementation of a privacy-preserving \ac{ml} platform. We started by describing the datasets chosen to evaluate the platform in Section \ref{sec:DatasetsImplementation}, and then we explained the preprocessing step in Section \ref{sec:DataPreProcessingImplementation}.
Section \ref{sec:BaselineImplementation} presented the implementation of the baseline for the chosen \ac{ml} algorithms, and Section \ref{sec:ExpandedAlgorithmsImplementation} detailed the implementation of the evaluation phase of those algorithms.
In Section \ref{sec:CryptoDomainImplementation}, we detailed which cryptographic protocols we used, why we used them, and how we implemented them, mentioning which toolkits were used.
For understanding the uses of the solution developed, we detailed in Section \ref{sec:usecaseHealthcare} a use case for our implementation, and finally in Section \ref{sec:components} we presented a possible usage for our implementation, in the form of a toolkit that could be used in a business environment.


  %
 %%%
%%%%%                        THE END
  %
  %%%%%%%%%%%%%%%%%%%%%%%%%%%%%%%%%%%%%%%%%%%%%%%%%%%%%%%%%%%%%%%%%%%%%%%%%%%%%
