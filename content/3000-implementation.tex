% -*- coding: utf-8 -*-
%

% reset all acronym expansions
\acresetall

%%%%%%%%%%%%%%%%%%%%%%%%%%%%%%%%%%%%%%%%%%%%%%
\chapter{Implementation}
\label{ch:Implementation}
%%%%%%%%%%%%%%%%%%%%%%%%%%%%%%%%%%%%%%%%%%%%%%

When building a platform in the scope of privacy-preserving \ac{ML}, we must consider not only the traditional steps in data processing mentioned in section \ref{sec:PrivacyImplicationsPersonalDataProcessing} and in Figure \ref{fig:crisp-dm}, but also have an increased care when preprocessing data to incorporate the cryptographic techniques that we used.
This chapter describes the work that was done in implementing a privacy-preserving \ac{ML} platform. We start off with a description of the datasets chosen to evaluate our platform \ref{sec:DatasetsImplementation}. Then, we explain the preprocessing that was done to those datasets \ref{sec:DataPreProcessingImplementation}.
Section \ref{sec:BaselineImplementation} presents the baseline implementation of the chosen \ac{ML} algorithms, resorting to a widely used \ac{ML} toolkit for Python.
In section \ref{sec:ExpandedAlgorithmsImplementation} the implementation of the prediction phase of the algorithms is detailed.
Finally, in section \ref{sec:CryptoDomainImplementation}, we detail which cryptographic protocols we used, why, and how we implemented them, mentioning which toolkits were used.



  %%%%%%%%%%%%%%%%%%%%%%%%%%%%%%%%%%%%%%%%%%%%%%%%%%%%%%%%%%%%%%%%%%%%%%%%%%%%%
  %
%%%%%                        THE BEGINNING
 %%%
  %


\input content/3010-implementation.tex
\input content/3020-use_case.tex


  %%%%%%%%%%%%%%%%%%%%%%%%%%%%%%%%%%%%%%%%%%%%%%%%%%%%%%%%%%%%%%%%%%%%%%%%%%%%%
  %
%%%%%                        LAST SECTION
 %%%
  %

  
%%%%%%%%%%%%%%%%%%%%%%%%%%%%%%%%%%%%%%%%%%%%%%
\section{Summary}
\label{sec:SummaryBARD}
%%%%%%%%%%%%%%%%%%%%%%%%%%%%%%%%%%%%%%%%%%%%%%


\todo[inline]{Summary chapter 3.}

  %
 %%%
%%%%%                        THE END
  %
  %%%%%%%%%%%%%%%%%%%%%%%%%%%%%%%%%%%%%%%%%%%%%%%%%%%%%%%%%%%%%%%%%%%%%%%%%%%%%
