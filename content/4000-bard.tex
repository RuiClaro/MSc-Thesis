% -*- coding: utf-8 -*-
%

% reset all acronym expansions
\acresetall

%%%%%%%%%%%%%%%%%%%%%%%%%%%%%%%%%%%%%%%%%%%%%%
\chapter{BARD}
\label{ch:BARD}
%%%%%%%%%%%%%%%%%%%%%%%%%%%%%%%%%%%%%%%%%%%%%%


In this chapter, we present the \ac{bard} project from Altran that this thesis is a part of. 
In Section \ref{sec:Motivation}, we present the motivations behind the creation of \ac{bard}, and define the objectives of it in Section \ref{sec:Objectives}.
Section \ref{sec:Architecture} presents the architectural specifications of the project.
Finally, since the development of this project was a team effort, Section \ref{sec:MyContributions} details our contributions to \ac{bard}.


  %%%%%%%%%%%%%%%%%%%%%%%%%%%%%%%%%%%%%%%%%%%%%%%%%%%%%%%%%%%%%%%%%%%%%%%%%%%%%
  %
%%%%%                        THE BEGINNING
 %%%
  %

%%%%%%%%%%%%%%%%%%%%%%%%%%%%%%%%%%%%%%%%%%%%%%%%%%%
\section{Motivation}
\label{sec:Motivation}
%%%%%%%%%%%%%%%%%%%%%%%%%%%%%%%%%%%%%%%%%%%%%%%%%%%

The idea for this project was motivated by the current growth trend in the Big Data market. The evolution of Big Data the last years, caused by the increasing number of devices connected to the Internet, provided analysts with the data to develop and improve systems in a varied scope of subjects, such as Healthcare, Automotive Industry, etc. But this data have private information about individuals, and, as we have shown before, processing them without certain precautions leads to breaches of private information. \ac{bard} project contributes to solving this problem, by raising awareness to it and providing solutions in the form of methods and protocols to build privacy-preserving solutions for Big Data systems.

%%%%%%%%%%%%%%%%%%%%%%%%%%%%%%%%%%%%%%%%%%%%%%%%%%%
\section{Objectives}
\label{sec:Objectives}
%%%%%%%%%%%%%%%%%%%%%%%%%%%%%%%%%%%%%%%%%%%%%%%%%%%

The objectives defined for \ac{bard} are described bellow:

\begin{itemize}
	
	\item Define methods to support protection of personal data for harvesting, sharing, querying and processing data assets, and supporting all the decisions to be taken while developing the platform.
	\item Analyze the effects of the current legislation on the validation of the solution proposed, as it may restrict which technical solutions can be used..
	\item Conceive a Privacy by Design architecture which makes the right balance between the data subject’s needs, the data consumer’s demands and the legal constraints.
	\item Develop a Privacy by Design platform based on a reference architecture for the entire data flow process, in order to maximize value for both the people and companies.
\end{itemize}

The expected output of \ac{bard} is a platform that provides mechanisms for the protection of personal data, that complies with the current legislation, and that assures Privacy by Design and by Default.


\input content/4010-architecture.tex

  %%%%%%%%%%%%%%%%%%%%%%%%%%%%%%%%%%%%%%%%%%%%%%%%%%%%%%%%%%%%%%%%%%%%%%%%%%%%%
  %
%%%%%                        LAST SECTION
 %%%
  %

  
%%%%%%%%%%%%%%%%%%%%%%%%%%%%%%%%%%%%%%%%%%%%%%
\section{Summary}
\label{sec:SummaryBARD}
%%%%%%%%%%%%%%%%%%%%%%%%%%%%%%%%%%%%%%%%%%%%%%

In this chapter, we discussed the project from Altran that this thesis is a part of.
We explained the motivations behind its creation in Section \ref{sec:Motivation}, and its objectives in Section \ref{sec:Objectives}.
We detailed the architecture of \ac{bard} in Section \ref{sec:Architecture}.
Finally, we explained our contributions to the project in Section \ref{sec:MyContributions}.


  %
 %%%
%%%%%                        THE END
  %
  %%%%%%%%%%%%%%%%%%%%%%%%%%%%%%%%%%%%%%%%%%%%%%%%%%%%%%%%%%%%%%%%%%%%%%%%%%%%%
