
%%%%%%%%%%%%%%%%%%%%%%%%%%%%%%%%%%%%%%%%%%%%%%%%%%%
\section{Platform}
\label{sec:components}
%%%%%%%%%%%%%%%%%%%%%%%%%%%%%%%%%%%%%%%%%%%%%%%%%%%



\begin{figure}[!h]
 \centering
  \includegraphics[width=0.5\textwidth]{images/conceptual_view_of_the_platform.pdf}
  \caption{Conceptual view of the platform.}
  \label{fig:ConceptView}
\end{figure}

In Figure \ref{fig:ConceptView} we present the conceptual view of our platform. 
The \emph{data resources} represent the datasets that are used in the classification process. 
The data processing itself is done using the combination of ML algorithms and cryptographic techniques for performing privacy-preserving computations. 
The Application Programming Interface (API) layer abstracts details and provides the operations of the platform itself, which allow a simplified building of applications and data visualizations.
The use-cases describe the various subjects that can be addressed using this platform, and allow us to place it in real-world scenarios that have high impact and demand in Big Data operations. More use-cases are possible beyond Healthcare, Mobility and Finance, as the platform is designed for general use.


Joining the algorithms and techniques described in the previous sections, we propose now the creation of a platform that would allow developers to create, implement, and expand the existing components. 


\begin{itemize}
  \item What \ac{ml} algorithm to use.
  \item What privacy-preserving technique to use.
  \item The dataset to process.
  \item Inspect the resulting data.
\end{itemize}



Finally, we now present how \acs{bard} could be implemented in a business scenario. The users, in this case, the companies that profit from using \acs{bard} to do knowledge learning, provide the dataset to train the model or select an existing model already trained with publicly available data, and provide the sample to be evaluated. A team of developers, that could be outsourced or part of the company, maintain the toolkit and develop new functionalities (new \ac{ml} algorithms, new privacy-preserving techniques, etc.). A central repository, that contains \ac{ml} models trained \textit{a priori} for different types of data context (healthcare, income, etc.). In Figure , we present a context diagram for \acs{bard}. 
