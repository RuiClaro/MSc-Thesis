
%%%%%%%%%%%%%%%%%%%%%%%%%
\section{Data Security} 
\label{sec:DataSecurity}
%%%%%%%%%%%%%%%%%%%%%%%%%


Data Security refers to protective digital measures that are applied to prevent unauthorized access to computers, databases, and websites, as well as to prevent data destruction or alteration.


%%%%%%%%%%%%%%%%%%%%%%%%%%%%%%%%%%%%
\subsection{Data Protection Goals}
\label{ssec:DataProtectionGoals}
%%%%%%%%%%%%%%%%%%%%%%%%%%%%%%%%%%%%


We define here the core protection goals widely accepted in the literature, often known as the CIA triad (\textit{Confidentiality}, \textit{Integrity}, and \textit{Availability}) \cite{Hansen2015}.

\begin{itemize}
    \setlength\itemsep{1em}

    \item \textit{Confidentiality} is defined as the property that data, and services that process such data, cannot be accessed by unauthorized entities.

    \item \textit{Integrity} is defined as the property that data, and services that process such data, cannot be modified in an unauthorized or undetected manner.

    \item \textit{Availability} is defined as the property that access to data, and services that process such data, is always possible when needed by the authorized parties and in a timely manner.

\end{itemize}


For applying Data Security measures, various technologies can be implemented, such as:

\begin{itemize}
    \setlength\itemsep{1em}

    \item \textit{Data backups} ensure that data that has been lost can be recovered. This technique is standard procedure for most companies since the permanent loss of crucial data can seriously cripple a company.

    \item \textit{Data erasure}, in contrast to backups, is a technique to permanently delete data from a hard drive or other digital media, to ensure that no sensitive data is leaked when a company wants to permanently remove an asset from usage or when required by court order.

    \item \textit{Data encryption}, or disk encryption, refers to techniques that allow a user to encrypt data in a disk or part of it, such that it remains protected and cannot be decrypted easily by an unauthorized party.

    \item \textit{Identity-based security} is a method to limit the access to data such that only a user that has been authenticated and has permission to access a piece of data can do so.


\end{itemize}


These techniques offer ways to protect data, but sometimes this is not enough. Usually due to incorrect programming that cause bugs in the system, vulnerabilities occur in the software that allows unauthorized parties to bypass these mechanisms and get access to data that should be confidential.



%%%%%%%%%%%%%%%%%%%%%%%%%%%%%%%%%%%%%%%%%%%%%%%%%    
\subsection{Examples of Data Security Breaches} 
\label{ssec:ExamplesDataSecurityBreaches}
%%%%%%%%%%%%%%%%%%%%%%%%%%%%%%%%%%%%%%%%%%%%%%%%%


Data Security breaches refer to attacks, usually by unauthorized access, to systems that contain private data. These attacks are commonly made by organized hacker groups to gain leverage against companies or to make a profit by selling the data in black markets. Next, we present some examples of recent Data Security breaches.
\begin{itemize}
    \setlength\itemsep{1em}

    \item Sony Pictures hack\footnote{\url{https://www.washingtonpost.com/news/the-switch/wp/2014/12/18/the-sony-pictures-hack-explained/}}. In 2014, a hacker group leaked confidential data from Sony Pictures in an attempt to gain leverage with the company to make it comply to their demands. The hacker group threatened to commit acts of terrorism in theaters if Sony released a movie related to the North Korean leader.

    \item Yahoo! data breach\footnote{\url{https://www.theguardian.com/technology/2016/dec/14/yahoo-hack-security-of-one-billion-accounts-breached}}. In 2016, Yahoo! reported two separate data breaches occurring in 2014 and 2013, of over 1.5 billion user accounts, including Yahoo! email access, which in turn can reveal bank and family details as well as passwords for other services.

    \item Ashley Madison data breach\footnote{\url{http://fortune.com/2015/08/26/ashley-madison-hack/}}. In 2015, a group of hackers stole user data from the adultery website Ashley Madison, and threatened to release usernames and personally identifying information if the website was not shut down.


\end{itemize}




