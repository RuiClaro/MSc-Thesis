% -*- coding: utf-8 -*-
%

% reset all acronym expansions
\acresetall

%%%%%%%%%%%%%%%%%%%%%%%%
\chapter{Introduction}
\label{ch:Introduction}
%%%%%%%%%%%%%%%%%%%%%%%%

  %%%%%%%%%%%%%%%%%%%%%%%%%%%%%%%%%%%%%%%%%%%%%%%%%%%%%%%%%%%%%%%%%%%%%%%%%%%%%
  %
%%%%%                        THE BEGINNING
 %%%
  %


With the so called ``Big Data revolution'', vast amounts of data are now being analyzed and processed by companies that take advantage of the enormous quantities of information that is generated every day\footnote{\url{http://www.vcloudnews.com/every-day-big-data-statistics-2-5-quintillion-bytes-of-data-created-daily/}}. Big Data and Business Analytics market reflects this growth rate, expecting to hit the \$210 billion mark in the year 2020\footnote{\url{https://www.idc.com/getdoc.jsp?containerId=prUS42371417}}.
Through this data processing, meaningful information can be obtained to improve existing systems or to discover new approaches in business models. An example of this is the deployment of \textit{Data Mining} algorithms to better understand their customers, and to devise better recommendation systems, in order to surpass their competitors in customer satisfaction. Another example lies in the field of healthcare, where it can be beneficial to match patient records from different hospitals in order to identify inefficiencies and develop best practices \cite{Lu2014}. 

Most times data contains private information about individuals, such as health records or daily routines. This kind of data cannot be freely processed because that leads to breaches of private information, such as the AOL Search Leak\footnote{\url{https://www.networkworld.com/article/2185187/security/15-worst-internet-privacy-scandals-of-all-time.html}} or the Microsoft Hotmail privacy breach\footnote{\url{https://www.networkworld.com/article/2185187/security/15-worst-internet-privacy-scandals-of-all-time.html}}. Due to these breaches, and despite the value that Data Mining adds to businesses and medical systems, consumers show an increasing concern in the privacy threats posed by it \cite{brankovic1999privacy}. The privacy of an individual may be violated due to, for example, unauthorized access to personal data, or the use of personal data for purposes other than the one for which data was collected.

To deal with the privacy issues in Data Mining, a sub-field known as \ac{ppdm} has been gaining influence over the last years \cite{DAcquisto2015}. The objective of PPDM is to guarantee the privacy of sensitive information, while at the same time preserve the utility of the data for Data Mining purposes \cite{agrawal2000privacy}.
This can be achieved by using one or more privacy-preserving techniques, such as \ac{dp} \cite{Danezis2015} or \ac{smpc} \cite{DAcquisto2015}.


\ac{ml} algorithms in the context of Big Data processing are also producing significant results, so that it is possible to do knowledge learning from datasets in order to predict future labels ( \textit{i.e.} classes of data) or clusters ( groups of related data) for new data. An example of an application of \ac{ml} algorithms in Data Mining is \textit{Classification} \cite{LeiXu2014}, in which a training set is processed in order to create a classifier for data, and then that classifier is used to predict class labels for new data. These applications show a greater impact in the field of medicine as mentioned above, with Google's DeepMind building \ac{ml} algorithms to process admissions in hospitals faster\footnote{\url{https://deepmind.com/applied/deepmind-health/}}, and with IBM's Watson supporting medical personnel consider treatment options for their patients\footnote{\url{https://www.mskcc.org/about/innovative-collaborations/watson-oncology}}.
Some examples of \ac{ml} algorithms include \ac{dt}, \ac{k-nn}, and \ac{svm} \cite{LeiXu2014}.

By combining \ac{ml} algorithms and privacy-preserving techniques, it is possible to create Data Mining processes that, not only allow for knowledge learning on large datasets but also to maintain a level of privacy that is desirable by individuals and that complies with the laws in force \cite{DAcquisto2015}.



\section{Contributions}
\label{sec:Intro_Contributions}

The main contribution of this thesis is the design and creation of a proof-of-concept platform for privacy-preserving distributed \ac{ml} computations without resorting to third parties. With it, we aim to give users a \ac{ml} platform without them having to build their own, meaning less maintenance costs derived of maintaining code developed or dedicated servers. Since the platform has its foundations on privacy-preserving techniques, it addresses satisfactorily the privacy demands that those users want for their data.
We show in \ref{sec:usecaseHealthcare} possible usages of this platform, detailing the context of the data to process, the trade-offs that are required, etc. We also present in section \ref{sec:components} the possible applications of the developed solution in a business environment.

We also provide in the evaluation chapter \ref{ch:Evaluation} a detailed comparison of four \ac{ml} algorithms (\ac{dt}, \ac{svm}, \ac{k-m} and \ac{lr}) combined with two privacy-preserving techniques (\ac{gc} and \ac{he}), allowing us to understand when to use each combination, depending on the context of data. 






%Write a first draft of this. Use what is mentioned in the abstract.
%Mention API and use cases.

  %%%%%%%%%%%%%%%%%%%%%%%%%%%%%%%%%%%%%%%%%%%%%%%%%%%%%%%%%%%%%%%%%%%%%%%%%%%%%
  %
%%%%%                        LAST SECTION
 %%%
  %

%%%%%%%%%%%%%%%%%%%%%%%%%%%%%%%%%%%%%%%%%{}
\section{Structure of this Document}
\label{sec:Intro_StructureOfThisDocument}
%%%%%%%%%%%%%%%%%%%%%%%%%%%%%%%%%%%%%%%%%
The remainder of this thesis is structured as follows.
In chapter \ref{ch:RelatedWork} we present an overview on the related work about the \ac{ppml} paradigm.
Chapter \ref{ch:BARD} we present the project from Altran that this thesis is a part of.
In chapter \ref{ch:Implementation} we discuss the implementation specifications of the project.
Chapter \ref{ch:Evaluation} presents and discuss the results obtained with the implementation.
Finally, in chapter \ref{ch:ConclusionsAndFutureWork} we wrap up the thesis with the conclusions and propose a continuation for this work.

  %
 %%%
%%%%%                        THE END
  %
  %%%%%%%%%%%%%%%%%%%%%%%%%%%%%%%%%%%%%%%%%%%%%%%%%%%%%%%%%%%%%%%%%%%%%%%%%%%%%
