% -*- coding: utf-8 -*-
%

% reset all acronym expansions
\acresetall

%%%%%%%%%%%%%%%%%%%%%%%%
\chapter{Introduction}
\label{ch:Introduction}
%%%%%%%%%%%%%%%%%%%%%%%%

  %%%%%%%%%%%%%%%%%%%%%%%%%%%%%%%%%%%%%%%%%%%%%%%%%%%%%%%%%%%%%%%%%%%%%%%%%%%%%
  %
%%%%%                        THE BEGINNING
 %%%
  %

\section{Motivation}
\label{sec:Intro_Motivation}

With the so called ``Big Data revolution'', vast amounts of data are now being analyzed and processed by companies. In this way, meaningful information can be obtained to improve existing systems or to discover new approaches in business models. It is common practice in most of these companies to deploy \textit{Data Mining} algorithms to better understand their customers, and to devise better recommendation systems, in order to surpass their competitors in customer satisfaction.
These practices are not only limited to customer satisfaction. In healthcare, for example, it can be beneficial to match patient records from different hospitals in order to identify inefficiencies and develop best practices \cite{Lu2014}.

Most times data contains private information about individuals, such as health records or daily routines. This kind of data cannot be freely processed because that goes against privacy laws and could lead to breaches of private information.
Despite the value of Data Mining results, people are showing an increasing concern relatively to the privacy threats posed by Data Mining \cite{brankovic1999privacy}. The privacy of an individual may be violated due to, for example, unauthorized access to personal data, or the use of personal data for purposes other than the one for which data was collected.

To deal with the privacy issues in Data Mining, a sub-field known as \ac{PPDM} has been gaining influence over the last years \cite{DAcquisto2015}. The objective of PPDM is to guarantee the privacy of sensitive information, while at the same time preserve the utility of the data for Data Mining purposes \cite{agrawal2000privacy}.
This can be achieved by using one or more privacy-preserving techniques, such as \ac{SMPC} \cite{DAcquisto2015} or \ac{DP} \cite{Danezis2015}.


Machine Learning algorithms in the context of Big Data processing are also producing significant results, so that it is possible to do knowledge learning from datasets in order to predict future labels or clusters for new data. An example of an application of Machine Learning algorithms in Data Mining is \textit{Classification} \cite{LeiXu2014}, in which a training set is processed in order to create a classifier for data, and then that classifier is used to predict class labels for new data. Some examples of Machine Learning algorithms include Decision Tree, \ac{k-NN}, and \ac{SVM} \cite{LeiXu2014}.


By combining Machine Learning algorithms and privacy-preserving techniques, it is possible to create Data Mining processes that, not only allow for knowledge learning on large datasets but also to maintain a level of privacy that is desirable by individuals and that complies with the laws in force.



\section{Contributions}
\label{sec:Intro_Contributions}

  %%%%%%%%%%%%%%%%%%%%%%%%%%%%%%%%%%%%%%%%%%%%%%%%%%%%%%%%%%%%%%%%%%%%%%%%%%%%%
  %
%%%%%                        LAST SECTION
 %%%
  %

%%%%%%%%%%%%%%%%%%%%%%%%%%%%%%%%%%%%%%%%%
\section{Structure of this Document}
\label{sec:Intro_StructureOfThisDocument}
%%%%%%%%%%%%%%%%%%%%%%%%%%%%%%%%%%%%%%%%%
The remainder of this thesis is structured as follows.

  %
 %%%
%%%%%                        THE END
  %
  %%%%%%%%%%%%%%%%%%%%%%%%%%%%%%%%%%%%%%%%%%%%%%%%%%%%%%%%%%%%%%%%%%%%%%%%%%%%%
