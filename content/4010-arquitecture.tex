
%%%%%%%%%%%%%%%%%%%%%%%%%%%%%%%%%%%%%%%%%%%%%%%%%%%
\section{Architecture}
\label{sec:Arquitecture}
%%%%%%%%%%%%%%%%%%%%%%%%%%%%%%%%%%%%%%%%%%%%%%%%%%%



This section presents the architectural specifications of \ac{BARD}.
\todo[inline]{Expand}


%%%%%%%%%%%%%%%%%%%%%%%%%%%%%%%%%%%%%%%%%%%%%%%%%%%
\subsection{Internal Structure}
\label{subsec:InternalStructure}
%%%%%%%%%%%%%%%%%%%%%%%%%%%%%%%%%%%%%%%%%%%%%%%%%%%

\todo[inline]{add a smallish intro to this section}

\ac{BARD} is composed of:

\begin{itemize}
	\setlength\itemsep{1em}
	\item A dataset to train the \ac{ML} algorithm, or the values representing the already trained algorithm.
	\item A sample or a set of samples that represent the input of the ``user'', to be predicted.
	\item A prediction algorithm that depends on the \ac{ML} algorithm and the privacy-preserving technique chosen.
	\item A set of toolkits for each of the techniques used.
	\item ...\todo{more?}
\end{itemize}



\missingfigure{BARD Scheme}
\commentPT{fazer um esquema para o BARD. não só de arquitectura, mas como as pessoas interagem com o BARD.


passos no flow geral de processamento de dados : recolha de dados, pre-processamento, etc
}

